% --------------------------------------------------------------------------- %
% Poster for the ECCS 2011 Conference about Elementary Dynamic Networks.      %
% --------------------------------------------------------------------------- %
% Created with Brian Amberg's LaTeX Poster Template. Please refer for the     %
% attached README.md file for the details how to compile with `pdflatex`.     %
% --------------------------------------------------------------------------- %
% $LastChangedDate:: 2011-09-11 10:57:12 +0200 (V, 11 szept. 2011)          $ %
% $LastChangedRevision:: 128                                                $ %
% $LastChangedBy:: rlegendi                                                 $ %
% $Id:: poster.tex 128 2011-09-11 08:57:12Z rlegendi                        $ %
% --------------------------------------------------------------------------- %
\documentclass[a0paper,portrait,amsmath,amssymb]{baposter}

\usepackage{relsize}		% For \smaller
\usepackage{url}			% For \url
\usepackage{epstopdf}	% Included EPS files automatically converted to PDF to include with pdflatex

\usepackage{bbm}
\usepackage{bm}
\usepackage[caption=false]{subfig}
\usepackage{floatrow}
\usepackage{amsmath}
\usepackage{enumitem}
\setlist{leftmargin=10pt, itemsep = 0.15em}
\usepackage{mdframed}   % for framing
\usepackage[font=small,labelfont=bf]{caption}
\usepackage{tcolorbox}

\usepackage[colorlinks,bookmarks=true,citecolor=blue,linkcolor=red,urlcolor=blue]{hyperref}
%%% Global Settings %%%%%%%%%%%%%%%%%%%%%%%%%%%%%%%%%%%%%%%%%%%%%%%%%%%%%%%%%%%

\graphicspath{{pix/}}	% Root directory of the pictures 
\tracingstats=2			% Enabled LaTeX logging with conditionals

%%% Color Definitions %%%%%%%%%%%%%%%%%%%%%%%%%%%%%%%%%%%%%%%%%%%%%%%%%%%%%%%%%

\definecolor{bordercol}{RGB}{40,40,40}
\definecolor{headercol1}{RGB}{186,215,230}
\definecolor{headercol2}{RGB}{80,80,80}
\definecolor{headerfontcol}{RGB}{0,0,0}
\definecolor{boxcolor}{RGB}{255,255,255}

%%%%%%%%%%%%%%%%%%%%%%%%%%%%%%%%%%%%%%%%%%%%%%%%%%%%%%%%%%%%%%%%%%%%%%%%%%%%%%%%
%%% Utility functions %%%%%%%%%%%%%%%%%%%%%%%%%%%%%%%%%%%%%%%%%%%%%%%%%%%%%%%%%%

%%% Save space in lists. Use this after the opening of the list %%%%%%%%%%%%%%%%
\newcommand{\compresslist}{
	\setlength{\itemsep}{1pt}
	\setlength{\parskip}{0pt}
	\setlength{\parsep}{0pt}
}

%%%%%%%%%%%%%%%%%%%%%%%%%%%%%%%%%%%%%%%%%%%%%%%%%%%%%%%%%%%%%%%%%%%%%%%%%%%%%%%
%%% Document Start %%%%%%%%%%%%%%%%%%%%%%%%%%%%%%%%%%%%%%%%%%%%%%%%%%%%%%%%%%%%
%%%%%%%%%%%%%%%%%%%%%%%%%%%%%%%%%%%%%%%%%%%%%%%%%%%%%%%%%%%%%%%%%%%%%%%%%%%%%%%

\begin{document}
\typeout{Poster rendering started}

%%% Setting Background Image %%%%%%%%%%%%%%%%%%%%%%%%%%%%%%%%%%%%%%%%%%%%%%%%%%
\background{
	%\begin{tikzpicture}[remember picture,overlay]%
	%\draw (current page.north west)+(-2em,2em) node[anchor=north west]
	%{\includegraphics[height=1.1\textheight]{background}};
	%\end{tikzpicture}
}

%%% General Poster Settings %%%%%%%%%%%%%%%%%%%%%%%%%%%%%%%%%%%%%%%%%%%%%%%%%%%
%%%%%% Eye Catcher, Title, Authors and University Images %%%%%%%%%%%%%%%%%%%%%%
\begin{poster}{
	grid=false,
	% Option is left on true though the eyecatcher is not used. The reason is
	% that we have a bit nicer looking title and author formatting in the headercol
	% this way
	%eyecatcher=false, 
	borderColor=bordercol,
	headerColorOne=headercol1,
	headerColorTwo=headercol2,
	headerFontColor=headerfontcol,
	% Only simple background color used, no shading, so boxColorTwo isn't necessary
	boxColorOne=boxcolor,
	headershape=roundedright,
	headerfont=\Large\sf\bf,
	textborder=rectangle,
	background=user,
	headerborder=open,
  boxshade=plain
}
%%% Eye Cacther %%%%%%%%%%%%%%%%%%%%%%%%%%%%%%%%%%%%%%%%%%%%%%%%%%%%%%%%%%%%%%%
{
	Eye Catcher, empty if option eyecatcher=false - unused
}
%%% Title %%%%%%%%%%%%%%%%%%%%%%%%%%%%%%%%%%%%%%%%%%%%%%%%%%%%%%%%%%%%%%%%%%%%%                                     
{\sf\bf
	Mesoscopic transport signatures of disorder-induced non-Hermitian phases
}
%%% Authors %%%%%%%%%%%%%%%%%%%%%%%%%%%%%%%%%%%%%%%%%%%%%%%%%%%%%%%%%%%%%%%%%%%
{
	Benjamin Michen$^1$, Jan Carl Budich$^1$\\
	{ Contact: benjamin.michen@tu-dresden.de}\\
	{\smaller $^1$ \emph{ Technische Universität Dresden, Institut für Theoretische Physik, 01062 Dresden}}\\

}
%%% Logo %%%%%%%%%%%%%%%%%%%%%%%%%%%%%%%%%%%%%%%%%%%%%%%%%%%%%%%%%%%%%%%%%%%%%%
{
% The logos are compressed a bit into a simple box to make them smaller on the result
% (Wasn't able to find any bigger of them.)

		\begin{minipage}{14em}
		\includegraphics[width = 12em]{CTQMAT_logo.png}\vspace{1em} \\
		\includegraphics[width = 12em]{TUD_logo.png}
		\end{minipage}
	
	}

\vspace{1em}
\headerbox{Abstract}{name=abstract,span = 2, column=0,row=0}{

We investigate the impact of disorder-induced {\bf exceptional points} (EPs) on the transport properties of a two-dimensional (2D) Dirac semi metal, where EPs refer to non-diagonalizable degeneracies of the effective non-Hermitian (NH) Hamiltonian $H_e(\bm k)$. We find that EPs may promote the  {\bf nearly vanishing conductance} of a finite sample at the Dirac point to a sizable value (see Fig.~\ref{FIG:illustration_setup}) that {\bf increases with disorder strength} (see Fig.~\ref{FIG:transport}) and also exhibits a strong {\bf directional anisotropy}. The exceptional phase also features quasiparticles with {\bf infinite lifetime} (see Fig.~\ref{FIG:spectrum}). 

\vspace{1em}
{\bf \large Preliminary Work}

Among other dissipative settings, the occurence of EPs has been demonstrated in a variety of disordered systems \cite{EPDisorder1, EPDisorder2, EPDisorder3, EP_Disorder_1D}. Up to now, possible experimental signatures of disorder-induced EPs have only been studied as a side note \cite{EP_Disorder_1D}.

}



\headerbox{Setup}{name=setup,span=1,column=2,row=0}{
(a)\\
{\includegraphics[trim={-2cm 16cm 4cm 1cm}, width = \linewidth]{sketch_setup_disorder_transport.png}}
(b)\\
{\includegraphics[trim={-1.7cm 2cm 1.7cm 0cm}, clip, width=\linewidth]{fig1b_dummy.png}}\label{illustration_setup_b}
(c)\\
{\includegraphics[trim={-1.7cm 1cm 1.7cm 0cm}, clip, width=\linewidth]{fig1c_dummy.png} }
\vspace{-2em}

\begin{minipage}{\linewidth}

\begin{tcolorbox}[width=\linewidth, colback=white!95!black, colframe=white, arc=3mm, boxrule=0mm, left=5pt,right=5pt]
\vspace{-0.5em}
\captionof{figure}{{\small (a) Illustration: Potential realization in the top layer of a so called square-net material \cite{SQN_Dirac_1}. (b) {\bf Orbital-restricted disorder} may split  the Dirac cones into topologically stable EPs with a characteristic square-root-like dispersion (absolute value of complex energy is shown). (c) For a sample of mesoscopic size, the vanishing zero-energy transmission seen in the clean regime (left panel) may be promoted to a finite value (right panel).} \label{FIG:illustration_setup}}
\vspace{-1.5em}
\end{tcolorbox}
\end{minipage}

%\includegraphics[angle=-90,width=0.98\linewidth]{PA_and_ER_Models_statisticalMeasures}
}


\headerbox{Complex Spectrum of the Exceptional Phase}{name=comspec, span = 2, column=0,below=abstract}{
\begin{minipage}{\linewidth}
{\includegraphics[trim={0cm 0cm 0cm 0.cm}, width=0.95\linewidth]{Spectrum_He.png}}\\
\vspace{-1.5em}
\begin{tcolorbox}[width=\linewidth, colback=white!95!black, colframe=white, arc=3mm, boxrule=0mm, left=5pt,right=5pt]
\vspace{-0.5em}
\captionof{figure}{\small Complex spectrum of $H_e(k)$ exhibiting EPs connected by {\bf Fermi arcs} (marked by red lines). Lines of {\bf vanishing imaginary part} (marked in green) entail propagating quasiparticles with {\bf infinite lifetime} and a group velocity parallel to the  $k_\mathrm{x}$-axis, resulting in {\bf anisotropic conductance}.} \label{FIG:spectrum}
\vspace{-1.5em}
\end{tcolorbox}
\vspace{0.5em}
\end{minipage}
}



\headerbox{Methodology}{name=meth, span = 2, column=0,below=comspec}{
\begin{itemize}
\item We consider a base model $H_0$ featuring a {\bf Dirac dispersion} with a random disorder term $V$.
\vspace{-0.5em}
\begin{align}
H_0 =&\sum_{\bm j}  \left[ t_\mathrm{x}  \psi_{\bm j+ \bm \delta_\mathrm{y}}^\dagger 
\sigma_\mathrm{x}  \psi_{\bm j} 
+ t_\mathrm{z}   \psi_{\bm j+ \bm \delta_\mathrm{x}}^\dagger \sigma_\mathrm{z}
 \psi_{\bm j} + \mathrm{h.c.} \right].  \label{H_0} \\
V = &\sum_{\bm j} a_{\bm j} \left[ \psi_{\bm j}^\dagger \left( 
s_0 \sigma_\mathrm{0} + \gamma \left(\sin(\phi) \sigma_\mathrm{x} + \cos(\phi)  \sigma_\mathrm{z} \right) \right) \psi_{\bm j} 
 + \mathrm{h.c.} \right ]. \label{EQN:disorder} 
\end{align}\vspace{-1.5em}

We set $t_x = t_y = 0.5$, $s_0, \gamma, \phi \in \mathbbm R$ and draw the uncorrelated random amplitudes $\{ a_{\bm j} \}$ from a box distribution on the real interval $[- \alpha, \alpha]$, so the overall disorder strength scales with $\alpha$. Generally, this parametrization allows for tuning between a uniform and an {\bf orbital-restricted} disorder potential.

\item Impurity scattering in the disordered system $H = H_0 + V$ is discribed through a NH {\bf self-energy correction} $\Sigma(\bm k, \omega)$ arising from a disorder-averaged Green's function. We take $H_e(\bm k) = H_0 + \Sigma(\bm k, \omega = 0)$ as the effective NH Hamiltonian, which will exhibit EPs for orbital-restricted disorder (see Fig.~\ref{FIG:spectrum}). 

\item We investigate the mesoscopic transport properties of our system by studying the {\bf linear response conductance} in a two-terminal setting (see Fig.~\ref{FIG:transport}).
\end{itemize}
}


\headerbox{References}{name=references,column=0, span = 2, above=bottom}{
\small 													% Make the whole text smaller
\vspace{-0.4em} 										% Save some space at the beginning
\bibliographystyle{plain}							% Use plain style
\renewcommand{\section}[2]{\vskip 0.05em}		% Omit "References" title
\begin{thebibliography}{1}							% Simple bibliography with widest label of 1
\itemsep=-0.01em										% Save space between the separation
\setlength{\baselineskip}{0.4em}					% Save space with longer lines

\bibitem{EPDisorder1}
A. A. Zyuzin, and A. Y. Zyuzin, {\em Flat band in disorder-driven non-Hermitian Weyl semimetals}, Phys. Rev. B {\bf 97}, 041203 (2018).

\bibitem{EPDisorder2}
K. Moors , A. A. Zyuzin, A. Y. Zyuzin, R. P. Tiwari, and T. L. Schmidt, {\em Disorder-driven exceptional lines
and Fermi ribbons in tilted nodal-line semimetals} Phys. Rev. B {\bf{99}}, 041116 (2019).

\bibitem{EPDisorder3}
A. A. Zyuzin, and P. Simon, {\em Disorder-induced exceptional points and nodal lines in Dirac superconductors}, Phys. Rev. B {\bf 99}, 165145 (2019).

\bibitem{EP_Disorder_1D}
B. Michen, T. Micallo, and J. C. Budich, {\em Exceptional non-Hermitian phases in disordered quantum wires}, Phys. Rev. B {\bf 104}, 035413 (2021).

\bibitem{SQN_Dirac_1}
S. Klemenz, S. Lei, L. M. Schoop, {\em Topological Semimetals in Square-Net Materials}, Annual Review of Materials Research {\bf 49} (1), 185-206  (2019).

\bibitem{EP_Disorder_2D}
B. Michen and J. C. Budich, {\em Mesoscopic transport signatures of disorder-induced non-Hermitian phases}, Phys. Rev. Research {\bf 4}, 023248 (2022).

\end{thebibliography}
}




\headerbox{Conductance From Disorder}
{name=cond_dis,span=1,column=2,below= setup }{
 \includegraphics[trim={0cm 0cm 0cm 0cm},clip, width= \linewidth]{Transport_alpha_Nx_100_Ny_200_Phi_0.png} 
\vspace{-0.7em}
\begin{minipage}{\linewidth}
\begin{tcolorbox}[width=\linewidth, colback=white!95!black, colframe=white, arc=3mm, boxrule=0mm, left=5pt,right=5pt]
\vspace{-0.5em}
\captionof{figure}{\small Zero-energy conductance in $x$-direction of a finite sample as a function of disorder strength $\alpha$. Disorder parameters from Eq.~(\ref{EQN:disorder})  are $s_0 = 1$, $\phi = 0$ for all curves, while we vary $\gamma = 1$, $0.9$ and $0.8$ to study deviations from perfectly orbital-restricted disorder ($\gamma=1$). } \label{FIG:transport}
\vspace{-1.5em}
\end{tcolorbox}
\end{minipage}
\vspace{-0.3em}
\begin{itemize}
\item In the exceptional phase, the {\bf conductance increases with disorder strength} $\alpha$ in the direction perpendicular to the Fermi arcs. 
\item The effect persists for small deviations from perfectly orbital-restricted disorder.
\end{itemize}
}

\headerbox{Summary and Outlook}
{name=sum,span=1,column=2, above = bottom}{
Disorder-induced EPs leave clear fingerprints on the transport signatures of a mesoscopic sample, most strikingly a disorder-induced boost of the conductance. 
A number of other interesting aspects are discussed in the full publication \cite{EP_Disorder_2D}: 
\vspace{-.5em}

\begin{itemize}
\item {\bf Directional anisotropy} of the conductance
\item {\bf Anomalous Anderson localization }
\item {\bf Orbital-restricted disorder} will generally permit the {\bf survival of Bloch modes} from the clean system (the green lines with zero imaginary energy in Fig.~\ref{FIG:spectrum}).
\end{itemize}
\vspace{-1em}
}
\end{poster}
\end{document}
